\documentclass[a4paper,12pt]{report}
\usepackage{graphicx}  % For including graphics
\usepackage{hyperref}  % For hyperlinks
\usepackage{listings}  % For code snippets
\usepackage{tocbibind} % For adding ToC to the table of contents
\usepackage{titlesec}  % For customizing titles

\titleformat{\chapter}[block]   % Set chapter format to block style (both on same line)
{\normalfont\huge\bfseries}     % Format for the chapter number and title
{\thechapter}                   % Shows the chapter number followed by a dot
{0.5em}                         % Space between the chapter number and title
{}                              % Formatting for the chapter title (leave empty)

\renewcommand{\contentsname}{Inhaltsverzeichnis}
\renewcommand{\listtablename}{Tabellenverzeichnis}
\renewcommand{\listfigurename}{Abbildungsverzeichnis}
\renewcommand{\lstlistlistingname}{Code-Ausschnitte}
\renewcommand{\abstractname}{Zusammenfassung}
\renewcommand{\chaptername}{Kapitel}

\begin{document}

% Title Page
    \begin{titlepage}
        \centering
        {\LARGE AI-Aided Caches-n-Logs Monitoring-n-Wipin}\\[1em]
        {\large Luca Scherer, Janic Scherer, Luca Ammann}\\[2em]
        \textit{BFH, Project 1 (BTI3031) 24/25}\\
        \textit{Betreuer: Simon Kramer}
        \vfill
        \textit{\today}
    \end{titlepage}

% Abstract
    \begin{abstract}
        \ldots
    \end{abstract}

% Table of Contents
    \tableofcontents
    \listoftables
    \listoffigures
    \lstlistoflistings

% Main Content


    \chapter{Einleitung}


    \section{Ausgangssituation}


    \section{Projektziel}


    \section{Prioritäten}


    \chapter{Spezifikation}


    \section{Systemabgrenzung}

    \subsection{Systemumgebung}

    \subsection{Prozessumgebung}

    \newpage
    \section{Anforderungen}
    \subsection{Funktionale Anforderungen}

    \begin{table}[h!]
        \centering
        \setlength{\leftmargini}{0.4cm}
        \begin{tabular}{|c|p{10cm}|}
            \hline
            \textbf{ID} & FR-001 \\ \hline
            \textbf{Anforderung} & Suche von Log und Cache Files \\ \hline
            \textbf{Beschreibung} & In einem bestimmten Verzeichnis durchsucht das System jegliche Dateinamen inkl. dieser in Unterordnern, um Log \& Cache Files zu identifizieren.  \\ \hline
            \textbf{Akzeptanzkriterien} &
            \begin{itemize}
                \item Der deamon kann via CLI gestartet werden.
                \item Alle Konfigurationen und Aktionen des deamon sind via CLI verfügbar.
            \end{itemize}
            \\ \hline
        \end{tabular}
        \caption{Funktionale Anforderung FR-001}\label{tab:table2}
    \end{table}


    \newpage
    \subsection{Grenz- und Vorbedingungen}
    TODO systemunabhängigkeit, evtl. erweiterbarkeit des ui (kommunikation zwischen cli und deamon),
    Folgend 2 eher nicht funktionale Anforderungen

    \begin{table}[h!]
        \centering
        \setlength{\leftmargini}{0.4cm}
        \begin{tabular}{|c|p{10cm}|}
            \hline
            \textbf{ID} & FR-001 \\ \hline
            \textbf{Anforderung} & Hintergrundprozess mit periodischem Task (deamon) \\ \hline
            \textbf{Beschreibung} & Das System muss einen Hintergrundprozess implementieren, der periodisch Aktionen durchführen kann. \\ \hline
            \textbf{Akzeptanzkriterien} &
            \begin{itemize}
                \item Nach dem Start des Prozesses, ist dieser für den Benutzer nicht mehr ersichtlich.
                \item Der Hintergrundprozess ist erweiterbar, sodass dieser periodisch und autonom Aktionen durchführen kann.
            \end{itemize}
            \\ \hline
        \end{tabular}
        \caption{Funktionale Anforderung FR-001}\label{tab:table}
    \end{table}

    \begin{table}[h!]
        \centering
        \setlength{\leftmargini}{0.4cm}
        \begin{tabular}{|c|p{10cm}|}
            \hline
            \textbf{ID} & FR-002 \\ \hline
            \textbf{Anforderung} & Benutzerschnittstelle via Konsole (CLI) \\ \hline
            \textbf{Beschreibung} & Mit dem laufenden deamon soll via Konsole interagiert werden können \\ \hline
            \textbf{Akzeptanzkriterien} &
            \begin{itemize}
                \item Der deamon kann via CLI gestartet werden.
                \item Alle Konfigurationen und Aktionen des deamon sind via CLI verfügbar
            \end{itemize}
            \\ \hline
        \end{tabular}
        \caption{Funktionale Anforderung FR-002}\label{tab:table2}
    \end{table}

    \section{Usability}

    \subsection{Personas}

    \subsection{Storyboard}

    \subsection{UX-Prototyping}


    \chapter{Implementierung}


    \section{Architektur}


    \section{Prozesse}
    \url{https://www.ganttlab.com} \\
    \url{https://www.hermes.admin.ch/de/projektmanagement/verstehen/ubersicht-hermes/methodenubersicht.html}


    \chapter{Bereitstellung/Integration}


    \section{Lizenzierung}


    \section{Installationshandbuch \& Skript}


    \section{Benutzerhandbuch}


    \chapter{Fazit}


    \section{Diskussion}


    \section{Zusammenfassung}


    \section{Zukünftige Arbeiten}
    \url{https://doi.org/10.1145/3664811}


    \chapter{Glossar}


    \chapter{Index}


    \chapter{Bibliografie}


    \chapter{Anhang}


    \section{Facsimile der Projektbeschreibung}


    \section{Erklärung zur Urheberschaft}

\end{document}
