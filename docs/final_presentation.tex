\documentclass[
    ngerman,%globale Übergabe der Hauptsprache
%	logofile=example-image, %Falls die Logo Dateien nicht vorliegen
    authorontitle=true,
]{bfhbeamer}


%\usepackage[main=ngerman]{babel}

% Der folgende Block ist nur bei pdfTeX auf Versionen vor April 2018 notwendig
\usepackage{iftex}
\ifPDFTeX
\usepackage[utf8]{inputenc}
\usepackage{blkarray}
\usepackage{graphicx}
\usepackage{biblatex}
\usepackage{amssymb}
\usepackage{amsmath}
\usepackage{mathtools}%kompatibilität mit TeX Versionen vor April 2018
\fi


%Makros für Formatierungen der Doku
%Im Allgemeinen nicht notwendig!
% \let\code\texttt

\title{TraceSentry}
\subtitle{AI-Aided Caches-n-Logs Monitoring-n-Wiping Daemon}
\author[L. Ammann, J. Scherer, L. Scherer]{Luca Ammann, Janic Scherer, Luca Scherer}
\date{8. Januar 2025}
\institute{BFH-TI}
\titlegraphic*{\includegraphics{assets/presentation/title}}%is only used with BFH-graphic and BFH-fullgraphic

%Activate the output of a frame number:
%\setbeamertemplate{page number in head/foot}[framenumber]

\begin{document}

    \setbeamertemplate{title page}[BFH-graphic]
    \maketitle

    \begin{frame}{Inhalt}
        \framesubtitle{Übersicht}
        \begin{columns}
            \begin{column}{0.4\textwidth}
                \tableofcontents
            \end{column}
            \begin{column}{0.6\textwidth}
                \begin{center}
                    %\includegraphics[width=0.3\textwidth]
                    %\includegraphics[width=0.6\textwidth]
                \end{center}
            \end{column}
        \end{columns}
    \end{frame}


    \section{Problemstellung}\label{sec:problemstellung}
    \setbeamertemplate{Problemstellung}[BFH-ruled]
    \frame{\sectionpage}

    \begin{frame}{Problemstellung}
        \framesubtitle{Problem}
        \begin{itemize}
            \item Cache- und Logdateien: tempor\"are Dateien von Betriebssystemen und Anwendungen.
            \item Viele Benutzer wissen nicht:
            \begin{itemize}
                \item Welche Dateien existieren.
                \item Welche Anwendungen diese manipulieren.
            \end{itemize}
            \item Risiken: Datenschutzprobleme und fehlende Kontrolle.
        \end{itemize}
    \end{frame}

    \begin{frame}{Problemstellung}
        \framesubtitle{Chance}
        \begin{itemize}

            \item Entwickeln einer Software f\"ur:
            \begin{itemize}
                \item \textbf{Suche} von Dateien.
                \item \textbf{Monitoring} von Dateien bzgl. deren Erstellung, Ver\"anderung und L\"oschung.
                \item \textbf{Analyse und Bewertung} von Dateien.
                \item \textbf{Bereinigung} von Dateien.
            \end{itemize}
            \item Nutzung eines KI-Modells f\"ur:
            \begin{itemize}
                \item Analyse von Struktur (Syntax) und Bedeutung (Semantik) von Dateien.
                \item Bewertung des Dateiinhalts bzgl.
                dessen Schädlichkeit.
                \item Empfehlungen von n\"otigen Massnahmen.
            \end{itemize}
        \end{itemize}
    \end{frame}

    \begin{frame}{Projektziel}
        \begin{itemize}
            \item Entwicklung einer \textbf{plattformunabh\"angigen, benutzerfreundlichen} Anwendung mit Dokumentation.
            \item Praxisnahe \textbf{Fallstudie} zur Evaluation des Potentials von aktuellen KI-Technologien.
            \item Fokus auf \textbf{Resultate} und Kernfunktionalit\"at.
        \end{itemize}
    \end{frame}

    \begin{frame}{Rahmenbedingungen}
        \begin{itemize}
            \item Entwicklung eines Hintergrundprozesses (\textbf{Daemon}).
            \item Kommandozeilen-Interface f\"ur Benutzerinteraktion (\textbf{CLI}).
            \item \textbf{Cache- und Log}dateien als prim\"arer Fokus, Erweiterung auf andere Dateitypen m\"oglich.
            \item \textbf{FLOSS-lizenzierte} und \textbf{plattformunabh\"angige} Software.
            \item Evaluation eines aktuellen KI-Modells. \textbf{Keine Eigenentwicklung}.
        \end{itemize}
    \end{frame}


    \section{Lösung}\label{sec:loesung}
    \setbeamertemplate{Lösung}[BFH-ruled]
    \frame{\sectionpage}

    \begin{frame}{Architektur}
        \begin{center}
            \includegraphics[width=0.7\textwidth]{assets/DeplDiagram}
        \end{center}
    \end{frame}

    \begin{frame}{Der Daemon}
        \framesubtitle{Spring basierter Service}
        \begin{columns}
            \begin{column}{0.7\textwidth}
                \begin{itemize}
                    \item \textbf{Periodisch \& automatisch:} Ausführung von Monitoringaufgaben
                    \begin{itemize}
                        \item Erstellen von Dateisystem-Snapshots (Hashbäume).
                        \item Vergleich von Snapshots (mit Vor- bzw.\ Nachfolger).
                    \end{itemize}
                    \item \textbf{Via REST-API:} Kommunikation mit Benutzer
                    \begin{itemize}
                        \item Suche von Dateien (z.B.\ mit regulären Ausdrücken).
                        \item Management von überwachten Pfaden.
                        \item Operationen auf Snapshots.
                        \item Anfragen von KI-Einschätzungen.
                        \item Löschen von Dateien.
                        \item Administrative Funktionen (Starten/Stoppen des Daemons).
                    \end{itemize}
                \end{itemize}
            \end{column}
            \begin{column}{0.3\textwidth}
                \begin{center}
                    \includegraphics[width=0.5\textwidth]{assets/presentation/ghost-smile}
                \end{center}
            \end{column}
        \end{columns}
        \note[item]{https://www.svgrepo.com/}

    \end{frame}

    \begin{frame}{Die Kommandozeilen}
        \framesubtitle{Spring Shell basierte CLI}
        \begin{columns}
            \begin{column}{0.7\textwidth}
                \begin{itemize}
                    \item \textbf{Warum eine CLI?}
                    \begin{itemize}
                        \item Effizienz bei der Entwicklung.
                        \item Zielgruppe: technisch versierte Benutzer.
                        \item Architektur würde in Zukunft auch eine GUI erlauben.
                    \end{itemize}
                    \item \textbf{Funktionalitäten:}
                    \begin{itemize}
                        \item Parsing und Validierung von Benutzereingaben.
                        \item Kommunikation mit dem Daemon via REST-API\@.
                    \end{itemize}
                \end{itemize}
            \end{column}
            \begin{column}{0.3\textwidth}
                \begin{center}
                    \includegraphics[width=0.5\textwidth]{assets/presentation/prompt}
                \end{center}
            \end{column}
        \end{columns}
        \note[item]{...}
    \end{frame}

    \begin{frame}{Datenbank}
        \framesubtitle{SQLite}
        \begin{columns}
            \begin{column}{0.7\textwidth}
                \begin{itemize}
                    \item File-basiert (serverlos)
                    \item Light-weight
                    \item Kompatibel mit Spring Data (JPA)
                    \begin{itemize}
                        \item JDBC-/Hibernate-Kompatibilität
                        \item Transaktionen
                        \item Grundlegende SQL-Funktionen
                        \item etc.
                    \end{itemize}
                \end{itemize}
            \end{column}
            \begin{column}{0.3\textwidth}
                \begin{center}
                    \includegraphics[width=1\textwidth]{assets/presentation/db}
                \end{center}
            \end{column}
        \end{columns}
    \end{frame}

    \begin{frame}{Monitoring (Hashb\"aume)}
        \begin{columns}
            \begin{column}{0.5\textwidth}
                \item Struktur:
                \begin{itemize}
                    \item Die \textbf{Wurzel} entspricht dem überwachten Pfad.
                    \item \textbf{Innere Knoten} entsprechen Unterverzeichnissen.
                    \item \textbf{Blätter} entsprechen Dateien (oder leeren Verzeichnissen).
                \end{itemize}
                \item Hashing:
                \begin{align*}
                    B &\coloneqq \text{Blatt} \\
                    N &\coloneqq \text{Nicht-Blatt} \\
                    K &\coloneqq \text{Kind} \\
                    H(B) &= H(Datei)\\
                    H(N) &= H(H(K_1) \|H(K_2) \| \dots \|H(K_k))\\
                \end{align*}
            \end{column}
            \begin{column}{0.5\textwidth}
                \begin{center}
                    \includegraphics[width=1\textwidth]{assets/presentation/hashtree}
                \end{center}
            \end{column}
        \end{columns}
    \end{frame}

    \begin{frame}{KI-Fallstudie}
    \end{frame}

    \section{Demo}\label{sec:demo}
    \setbeamertemplate{Demo}[BFH-ruled]
    \frame{\sectionpage}

    \begin{frame}{Demo Frame}
        \framesubtitle{Untertitel}
        Bla Bla
    \end{frame}

    \section{Projektmanagement}\label{sec:projektmanagement}
    \setbeamertemplate{Projektmanagement}[BFH-ruled]
    \frame{\sectionpage}

    \begin{frame}{Rückblick}
        \framesubtitle{Produktziele}
        {
            \centering
            \begin{tabular}{l|c}
                \textbf{Ziel}                & \textbf{Erreicht} \\
                \hline
                Plattformunabhängig          & Ja                \\
                Suche von Dateien            & Ja                \\
                Periodisches Monitoring      & Ja                \\
                Datei-Bereinigung            & Ja                \\
                Modular und erweiterbar      & Ja                \\
                Automatische Testabdeckung   & Ja                \\
                AI-Einbindung mit Fallstudie & Ja                \\
            \end{tabular}\par
        }

        % todo: Product / Sprint Backlog Management, mby Burndown-Chart
    \end{frame}

    \begin{frame}{Sprintziele}
        \framesubtitle{Nach SMART definiert}
        Ziele der ersten 2 Sprints:
        \begin{columns}
            \column{0.7\textwidth}
            \begin{enumerate}
                \item Aufsetzen von lauffähiger CLI- und Daemon-Komponente und
                vollständige, sowie getestete Implementation der Search-Funktion (gemäss Testkonzept).
                \item Monitoring-Kommandos (gemäss Spezifikation) für das Persistieren
                und Auflisten der Dateipfade sind vollständig implementiert und
                getestet (gemäss Testkonzept).
            \end{enumerate}

            \column{0.3\textwidth}
            \textbf{S}pecific\\
            \textbf{M}easurable\\
            \textbf{A}chievable\\
            \textbf{R}elevant\\
            \textbf{T}ime-bound
        \end{columns}

    \end{frame}

    \begin{frame}{Einrichtung}
        \begin{enumerate}
            \item Erfassung Ausganssituation
            \begin{itemize}
                \item Proposal und Mail von Herr Kramer gelesen und im Team besprochen
                \item Kick-Off Meeting im Team mit Herr Kramer
            \end{itemize}
            \item Themen-Analyse
            \begin{enumerate}
                \item PO hat Epics für Grundfunktionalitäten erstellt, soweit dass die Anforderungen vom Proposal abgedeckt werden konnten.
                \item Epics sinnvoll priorisiert (nach Wichtigkeit und praktischer Umsetzbarkeit)
            \end{enumerate}
            \item Organisation
            \begin{itemize}
                \item SCRUM-Management in GitLab (Board, Epics, Issues, Iterations)
                \item Kommunikation im Team über Teams/Discord und Whatsapp.
                Mit den Betreuern über Mail und BigBlueButton.
            \end{itemize}
        \end{enumerate}
    \end{frame}

    \begin{frame}{Retro 1}
        \framesubtitle{Erfahrungen bezüglich SCRUM-Rollen und Stakeholdern}
        \begin{columns}
            \column{0.5\textwidth}
            \begin{table}[h!]
                \centering
                \begin{tabular}{p{3cm}|p{3.5cm}}
                    \textbf{Rolle} & \textbf{Teammitglied} \\
                    \hline
                    Product Owner  & Luca Scherer          \\
                    Scrum Master   & Luca Ammann           \\
                    Developer      & Team                  \\
                \end{tabular}
            \end{table}
            \smallskip
            \begin{table}[h!]
                \centering
                \begin{tabular}{p{3cm}|p{3.5cm}}
                    \textbf{Rolle} & \textbf{Name}    \\
                    \hline
                    Betreuer       & Dr. Simon Kramer \\
                    PM-Betreuer    & Frank Helbling   \\
                \end{tabular}
            \end{table}

            \column{0.5\textwidth}

            \vspace{0.5cm}

            => Arbeitslast gut aufgeteilt und konnten gegenseitig bei den Rollen unterstützen.

            \vspace{2cm}

            => Angenehm mit weniger Stakeholdern, da es weniger Kommunikationsaufwand gibt.
        \end{columns}
    \end{frame}

    \begin{frame}{Retro 2}
        \framesubtitle{Erfahrungen bezüglich Scrum Events und Artifacts}
        \begin{columns}
            \column{0.5\textwidth}
            \textbf{Events:}
            \begin{itemize}
                \item \textbf{Retro}: wurde bereits jederzeit im Team gelebt
                \item \textbf{Review}: verlief sehr effizient durch paralleles Erklären und Anschauen der MRs im ganzen Team
                \item \textbf{Planning}: ebenfalls effizient dadurch, dass Issues schon von PO vorbereitet waren,
                also konnte er sie direkt vorstellen, dann im Team besprechen, schätzen, zuweisen und Sprint-Ziel definieren.
            \end{itemize}

            \column{0.5\textwidth}
            \textbf{Artifacts:}
            \begin{itemize}
                \item \textbf{Board/Sprint}: war manchmal schwierig zu entscheiden, ob es in den Sprint kommt wenn es "nur" eine administrative Story ist.
                \item \textbf{Velocity}: war schwierig konstant zu halten und Commitment war eher nach Gefühl.
                \item \textbf{Subtasks}: eher overhead.
                \item \textbf{Burndown-Chart}: nicht wirklich hilfreich.
            \end{itemize}
        \end{columns}
    \end{frame}

    \begin{frame}{Retro 3}
        \framesubtitle{Erkenntnisse zu verwendeten Tools}
        \begin{itemize}
            \item \textbf{Discord/Teams}: vor allem Gruppencall mit ScreenSharing genutzt (nichts vermisst).
            \item \textbf{GitLab}: etwas gewöhnungsbedürftig zu Beginn, im Vergleich zu Jira schon um einiges eingeschränkter aber reichte für unseren Rahmen gut aus.
            (nichts vermisst)
            \item \textbf{Latex}: war wichtig für uns einmal anzuschauen und wahrscheinlich konnten alle daraus am meisten lernen in diesem Projekt, was für Zukunft sehr hilfreich ist.
            Braucht schon immer wieder Zeit um gewisse Dinge herauszufinden wie es in Latex gemacht werden kann aber lohnt sich.
            Sehr praktisch durch "docs as code" um über git zusammenzuarbeiten und es mit dem Projekt zusammen zu versionieren.
        \end{itemize}
    \end{frame}

    \section{Fazit}\label{sec:fazit}
    \setbeamertemplate{Fazit}[BFH-ruled]
    \frame{\sectionpage}

    \begin{frame}{Lessons learned}
        \framesubtitle{Perspektive des Teams}
        \begin{itemize}
            \item Erkenntnisse Rahmenbedingungen
            \begin{itemize}
                \item ein bisschen mühsam immer Termine zu finden die für alle passen
                \item eher schwierig die gewünschte Zeit zu investieren
                \item positiv, dass wir bei allem recht "frei" waren
                \item durch den Rahmen auch recht schwierig Scrum diszipliniert/konstant anzuwenden
            \end{itemize}
            \item Erkenntnisse Zusammenarbeit (im Team, Fachbetreuer, PM-Coach)
            \begin{itemize}
                \item Betreuer waren immer zur Verfügung für alle Anfragen per Mail und Herr Kramer auch für persönliche Besprechungen im Projektzeitfenster und antworteten hilfreich.
                \item Betreuer schränkte gar nicht ein bezüglich Entscheidungen sondern gab hilfreiche Tipps.
                \item Zusammenarbeit im Team hat sehr gut geklappt, dank dem, dass wir uns schon lange kennen harmonieren wir dementsprechend auch schon von Anfang an gut.
                \item Dadurch, dass wir uns auch oft neben dem Projektmodul treffen konnten wir auch oft nebenbei uns kurz darüber austauschen.
            \end{itemize}
        \end{itemize}
    \end{frame}

    \begin{frame}{Fazit zu Software & KI-Potenzial}
        \framesubtitle{Untertitel}
        Haben wir unsere Fragen beantwortet?
        Ziele erreicht?
    \end{frame}

    \section{Ausblick}\label{sec:ausblick}
    \setbeamertemplate{Ausblick}[BFH-ruled]
    \frame{\sectionpage}

    \begin{frame}{Ausblick Frame}
        \framesubtitle{Untertitel}
        Bla Bla
    \end{frame}

    \begin{frame}{Fragen Frame}
        \framesubtitle{Untertitel}
        Bla Bla
    \end{frame}


    \section{examples}\label{sec:examples}
    \begin{frame}{Blocks}
        \begin{block}{Block with a title}
            Content.
        \end{block}
        \begin{block}{}
            Without title
        \end{block}
    \end{frame}

    \begin{frame}{Block types}
        \begin{exampleblock}{Exampleblock}
            Content.
        \end{exampleblock}
        \begin{alertblock}{Alertblock}
            Content.
        \end{alertblock}
        \begin{example}[Example environment]
            Content.
        \end{example}
    \end{frame}

\end{document}
