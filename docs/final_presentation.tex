\documentclass[
	ngerman,%globale Übergabe der Hauptsprache
%	logofile=example-image, %Falls die Logo Dateien nicht vorliegen
	authorontitle=true,
	]{bfhbeamer}


%\usepackage[main=ngerman]{babel}

% Der folgende Block ist nur bei pdfTeX auf Versionen vor April 2018 notwendig
\usepackage{iftex}
\ifPDFTeX
\usepackage[utf8]{inputenc}
\usepackage{blkarray}
\usepackage{graphicx}%kompatibilität mit TeX Versionen vor April 2018
\fi


%Makros für Formatierungen der Doku
%Im Allgemeinen nicht notwendig!
\let\code\texttt

\title{TraceSentry}
\subtitle{AI-Aided Caches-n-Logs Monitoring-n-Wiping Daemon}
\author[L. Ammann, J. Scherer, L. Scherer]{Luca Ammann, Janic Scherer, Luca Scherer}
\date{8. Januar 2025}
\institute{BFH-TI}
\titlegraphic*{\includegraphics{assets/presentation/title}}%is only used with BFH-graphic and BFH-fullgraphic

%Activate the output of a frame number:
%\setbeamertemplate{page number in head/foot}[framenumber]

\begin{document}

\setbeamertemplate{title page}[BFH-graphic]
\maketitle

\begin{frame}{Inhalt}
	\framesubtitle{Übersicht}
	\begin{columns}
		\begin{column}{0.4\textwidth}
			\tableofcontents
		\end{column}
		\begin{column}{0.6\textwidth}
			\begin{center}
				%\includegraphics[width=0.3\textwidth]
				%\includegraphics[width=0.6\textwidth]
			\end{center}
		\end{column}
	\end{columns}
\end{frame}


\section{Problemstellung}\label{sec:problemstellung}
\setbeamertemplate{Problemstellung}[BFH-ruled]
\frame{\sectionpage}

\begin{frame}{Problemstellung}
	\framesubtitle{Problem}
	\begin{itemize}
		\item Cache- und Logdateien: tempor\"are Dateien von Betriebssystemen und Anwendungen.
		\item Viele Benutzer wissen nicht:
		\begin{itemize}
			\item Welche Dateien existieren.
			\item Welche Anwendungen diese manipulieren.
		\end{itemize}
		\item Risiken: Datenschutzprobleme und fehlende Kontrolle.
	\end{itemize}
\end{frame}

\begin{frame}{Problemstellung}
	\framesubtitle{Chance}
	\begin{itemize}

		\item Entwickeln einer Software f\"ur:
		\begin{itemize}
			\item \textbf{Suche} von Dateien.
			\item \textbf{\"Uberwachung} von Dateien bzgl. deren Erstellung, Ver\"anderung und L\"oschung.
		\end{itemize}
		\item Nutzung eines KI-Modells f\"ur:
		\begin{itemize}
			\item Analyse von Struktur (Syntax) und Bedeutung (Semantik) von Dateien.
			\item Bewertung des Dateiinhalts bzgl.
			dessen Schädlichkeit.
			\item Empfehlungen f\"uer n\"otige Massnahmen.
		\end{itemize}
	\end{itemize}
\end{frame}

\begin{frame}{Problemstellung}
	\framesubtitle{Antrag}
	\begin{itemize}
		\item Entwicklung einer Software f\"ur:
		\begin{itemize}
			\item \textbf{Suche, \"Uberwachung und L\"oschung} von Dateien.
			\item Einholen von Einsch\"atzungen eines KI-Modells.
		\end{itemize}
		\item Stakeholder:
		\begin{itemize}
			\item Simon Kramer (Betreuer und Auftraggeber).
		\end{itemize}
	\end{itemize}
\end{frame}

\begin{frame}{Projektziel}
	\begin{itemize}
		\item Entwicklung einer plattformunabh\"angigen, benutzerfreundlichen Anwendung mit Dokumentation.
		\item Funktionen:
		\begin{itemize}
			\item \textbf{Suche, \"Uberwachung und L\"oschung} von Dateien.
			\item Analyse und Bewertung durch ein eingebundenes KI-Modell.
		\end{itemize}
		\item Abgrenzung:
		\begin{itemize}
			\item Keine Eigenentwicklung des KI-Modells, nur Einbindung und Nutzung.
		\end{itemize}
	\end{itemize}
\end{frame}

\begin{frame}{Priorit\"aten}
	\begin{itemize}
		\item Fokus auf Resultate, nicht auf Prozesse.
		\item Fertiges Projekt statt unfertige Zwischenprodukte.
		\item \textbf{MVP (Minimum Viable Product):}
		\begin{itemize}
			\item Kernfunktionalit\"at:
			\begin{itemize}
				\item Suche und \"Uberwachung von Dateien.
				\item Snapshots erstellen und vergleichen.
				\item KI-Einsch\"atzungen einholen.
			\end{itemize}
			\item Wichtig: Dokumentation, Tests, Codequalit\"at.
		\end{itemize}
		\item Zur\"uckgestellte Erweiterungen: siehe \textit{Zuk\"unftige Arbeiten}.
	\end{itemize}
\end{frame}



\section{Lösung}\label{sec:loesung}
\setbeamertemplate{Lösung}[BFH-ruled]
\frame{\sectionpage}

\begin{frame}{Lösung Frame}
	\framesubtitle{Untertitel}
	Bla Bla
\end{frame}

\section{Demo}\label{sec:demo}
\setbeamertemplate{Demo}[BFH-ruled]
\frame{\sectionpage}

\begin{frame}{Demo Frame}
	\framesubtitle{Untertitel}
	Bla Bla
\end{frame}

\section{Projektmanagement}\label{sec:projektmanagement}
\setbeamertemplate{Projektmanagement}[BFH-ruled]
\frame{\sectionpage}

\begin{frame}{Projektmanagement Frame}
	\framesubtitle{Untertitel}
	Bla Bla
\end{frame}

\section{Fazit}\label{sec:fazit}
\setbeamertemplate{Fazit}[BFH-ruled]
\frame{\sectionpage}

\begin{frame}{Fazit Frame}
	\framesubtitle{Untertitel}
	Bla Bla
\end{frame}

\section{Ausblick}\label{sec:ausblick}
\setbeamertemplate{Ausblick}[BFH-ruled]
\frame{\sectionpage}

\begin{frame}{Ausblick Frame}
	\framesubtitle{Untertitel}
	Bla Bla
\end{frame}

\begin{frame}{Fragen Frame}
	\framesubtitle{Untertitel}
	Bla Bla
\end{frame}

\section{examples}\label{sec:examples}
\begin{frame}{Blocks}
	\begin{block}{Block with a title}
		Content.
	\end{block}
	\begin{block}{}
		Without title
	\end{block}
\end{frame}

\begin{frame}{Block types}
	\begin{exampleblock}{Exampleblock}
		Content.
	\end{exampleblock}
	\begin{alertblock}{Alertblock}
		Content.
	\end{alertblock}
	\begin{example}[Example environment]
		Content.
	\end{example}
\end{frame}

\end{document}

